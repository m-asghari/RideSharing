\vspace{-2mm}
\section{Related Work}
\label{sec:related}
%In the following, we first review the related work regarding static ride-sharing, then discuss real-time ride-sharing. Pricing scheme and task assignment and scheduing in spatial crowdsouring.

% offline ride-sharing
There are mainly two categories of ride-sharing, i.e., static and dyanmic ride-sharing. Most existing studies~\cite{FuruhataTRB13, Santi14, CiciUbicom14} focus on static ride-sharing, where all riders and drivers are known a priori and thus, trips are prearranged. Furuhata et al.~\cite{FuruhataTRB13} provoides a comprehensive suvery of the different types of ride-sharing regarding their formulations, optimizations and key computation challenges. Santi et al.~\cite{Santi14} proposes a graph-based approach to quantify the potential of ride-sharing using New York's taxi data, and Cici et al.~\cite{CiciUbicom14} evaluated the potential of carpooling using four cities' mobile dataset. In addition, ride-sharing problem can be treated as a special class of the dial-a-ride problem (DARP)~\cite{Cordeau07}, or dynamic vehicle routing problem (VRP)~\cite{Dantzig59, LiSstd15} in operational research, which is proven to be NP-hard. All these studies assume that the riders' and drivers' statuses are know in advance, and hence can afford high computation cost, which is not the case in real-time ride-sharing.

% real-time ride-sharing
<<<<<<< HEAD
With the emergence of many ridesahring mobile applications (e.g., Uber and Lyft), real-time ride-sharing~\cite{Ma13, Ma15, Huang14,Ota15, Cici15, Cao15, PelzerITS15} has recently attracted more research interest. Ma et al.~\cite{Ma13, Ma15} proposed a ride-sharing dispatch system named ``T-share" to serve the rider request on-the-fly with the objective of reducing drivers' total travel distance. Their work focuses on maintaining a spatial-temporal index to retrieve the candidate drivers. On the other hand, Huang~\cite{Huang14} proposed a kinect tree scheduling algorithm to dynamically match trip request to drivers with minimum incurred travel distance. Ota et al.~\cite{Ota15} introduced a data-driven simuation framework that enables the analysis of ride-sharing by using New York's taxi dataset. Santos et. al~\cite{SantosIjcai13} propose a ride-sharing system to maximize the number of matched request. The majority of these studies aim to minimize the total travel distance of drivers, however, we show that this does not necessarily mean shorter travel distance for the riders. Compared with these work, we discuss the conflicting interest between riders, drivers and platform providers. We propose a general and versatile pricing model and  our objective is to maximize the total profit of the platform provider. We show that by maximizing the overall profit, our framework achieves higher service rate and quality.  Finally, we introduced a decentralized auction-based framework to support scalable and real-time scheduling, which differs from existing centralized scheduling framework. 
=======
With the emergence of many ridesahring mobile applications (e.g., Uber and Lyft), real-time ride-sharing~\cite{Ma13, Ma15, Huang14,Ota15, Cici15, Cao15, PelzerITS15} has recently attracted more research interest. Ma et al.~\cite{Ma13, Ma15} proposed a ride-sharing dispatch system named ``T-share" to serve the rider request on-the-fly with the objective of reducing drivers' total travel distance. Their work focuses on maintaining a spatial-temporal index to retrieve the candidate drivers. On the other hand, Huang~\cite{Huang14} proposed a kinect tree scheduling algorithm to dynamically match trip request to drivers with minimum incurred travel distance. Ota et al.~\cite{Ota15} introduced a data-driven simuation framework that enables the analysis of ride-sharing by using New York's taxi dataset. Santos et. al~\cite{SantosIjcai13} propose a ride-sharing system to maximize the number of matched request. The majority of these studies aim to minimize the total travel distance of drivers, however, we show that this does not necessarily mean shorter travel distance for the riders. Compared with these work, we discuss the conflicting interest between riders, drivers and platform providers. We propose a general and versatile pricing model and  our objective is to maximize the total profit of the platform provider. We show that by maximizing the overall profit, our framework achieves higher service rate and quality.  Finally, We introduced a decentralized auction-based framework to support scalable and real-time scheduling, which differs from existing centralized scheduling framework. 
>>>>>>> 46f67d188a36b6f09f72c25697c415296ddf3444
%Because of this decentralized framework, the kinect tree scheduling algorithm proposed in Huang~\cite{Huang14} can be easily integrated into APART to purse further efficiency. \dingedit{Do we need this last sentence?}

% pricing scheme
Pricing mechanisms in realtime ride-sharing have also been studied in~\cite{KamarIJCAI09,KleinerIJCAI11, ZhaoAAMAS14}. Their main focus is to adapt the well-known VCG mechanism~\cite{Nisan07} for truthful bidding, while addressing the specific challenges such as computational issues, incentive compatibility~\cite{KamarIJCAI09, KleinerIJCAI11} and deficit control~\cite{ZhaoAAMAS14}. These studies are orthogonal to our paper, and can be integrated into APART when we compute the bid of each candidate driver. 

% spatial crowdsourcing
Our work is also related to the task assignment and scheduling problem in spatial crowdsourcing~\cite{KazemiGis12, DengGis13, DengGis15}. Spatial crowdsourcing is a platform, which enables a requester to comission workers to physically travel to some specified locations to perform a set of spatial tasks. For example, Kazemi et. al~\cite{KazemiGis12} formuated task assignment in spatial crowdsourcing as a min-cost max-flow problem, Deng et. al studied both task scheduling problem for one single worker~\cite{DengGis13} and multiple workers~\cite{DengGis15}. Unlike spatial crowdsourcing, in real-time ride-sharing, each rider request consists of both pickup and dropoff locations. In addition, their assignment and scheduling are processed in a batch fashion (e.g., batching tasks and workers every 10 minutes), whereas each rider request in real-time ride-sharing must be processed in a short amount of time.
\vspace{-0.1in}