\section{Related Work}
%In the following, we first review the related work regarding static ridesharing, then discuss real-time ridesharing. Pricing scheme and task assignment and scheduing in spatial crowdsouring.

% offline ridesharing
There are mainly two categories of ridesharing, i.e., static and dyanmic ridesharing. Most existing studies~\cite{FuruhataTRB13, Santi14, CiciUbicom14} belong to static ridesharing, where all riderS and drives are knownn in priori, and thus trips are prearranged. Furuhata et al.~\cite{FuruhataTRB13} provoides a compresensive suvery of the different types of ride sharing regarding their formulations, optimizations and key computations challenges. Santi et al.~\cite{Santi14} proposed a graph-based approach to quantify the potential of ridesharing using NewYork taxi data, and Cici et al.~\cite{CiciUbicom14} evaluated the potential of carpooling using four cities' mobile dataset. In addition, ridesharing problem can be treated as a special class of the dial-a-ride problem (DARP)~\cite{Cordeau07}, or dynamic vehicle routing problem (VRP)~\cite{LiSstd15} in operational research, which is proven to be NP-hard. All these studies assume that the rider and drivers status are know in advance, and hence can afford high computation cost, which is not the case in real-time ridesharing.

% real-time ridesharing
With the emergence of many ridesahring mobile applications (e.g., Uber and Lyft), real-time ridesharing~\cite{Ma13, Ma15, Huang14,OtaBigdata15, CiciGis15, CaoMDM15, PelzerITS15} attracts more research interest recently. Ma et al.~\cite{Ma13, Ma15} proposed a ridesharing dispatch system named ``T-share" to serve the rider request on-the-fly with the objective of reducing driver's total travel distance. Their work focus on maintaining a spatial-temporal indexing to retrieve the candidate drivers. On the other hand, Huang~\cite{Huang14} proposed a kinect tree scheduling algorithm to dynamically match trip request to drivers with minimum incurred travel distance. Ota et al.~\cite{OtaBigdata15} introduced a data-driven simuation framework that enables the analysis of ride-sharing by using NY taxi dataset. Santos et. al~\cite{SantosIjcai13} propose a ride sharing system to maximize the number of matched request. The majority of these studies aim to minimize the total travel distance of drivers, however, we show that this does not necessarily mean shorter travel distance for the riders. Compared with these work, we discuss the conflicting interest between riders, drivers and platform providers. We propose a general and flexible pricing scheme and  our objective is to maximize the total profit of platform. We show that by maximizing the overall profit, our framework achieves higher service rate and quality.  Finally, We introduced a decentralized auction-based framework to support scalable and real-time scheduling, which differs existing centralized scheduling framework. Because of this decentralized framework, the kinect tree scheduling algorithm proposed in Huang~\cite{Huang14} can be easily integrated into APART to purse further efficiency. \dingedit{Do we need this last sentence?}

% pricing scheme
Pricing mechanisms in realtime ridesharing have also been studied in~\cite{KamarIJCAI09,KleinerIJCAI11, ZhaoAAMAS14}. Their main focus is to adapt the well-known VCG mechanism~\cite{Nisan07} for truthful bidding, while addressing the specific challenges such as computational issues, incentive compatibility~\cite{KamarIJCAI09, KleinerIJCAI11} and deficit control~\cite{ZhaoAAMAS14}. These studies are orthogonal to our paper, and can be integrated into our framework APART when we compute the bid of each candidate driver. 

% spatial crowdsourcing
Our work is also related to the task assignment and scheduling problem in spatial crowdsourcing~\cite{KazemiGis12, DengGis13, DengGis15}. Spatial crowdsourcing is a platform, whic enables a requester to comission workers to physically travel to some specified locations to perform a set of spatial tasks. For example, Kazemi et. al~\cite{KazemiGis12} formuated task assignment in spatial crowdsourcing as a min-cost max-flow problem, Deng et. al studied both task scheduling problem for one single worker~\cite{DengGis13} and multiple workers~\cite{DengGis15}. Different with spatial crowdsourcing, in real-time ridesharing, each rider request consists of both pickup and dropoff locations. In addition, their assignment and scheduling are processed in a batch fashion (e.g., batching tasks and workers every 10 minutes), whereas each rider requst in realtime ridesharing must be processed in a short amount of time.
