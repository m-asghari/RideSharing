\section{Problem Definition}
\label{sec:problem_def}

In this section, we define the terminologies, and formally define the problem under consideration.

\subsection{Basic Concepts}
The road network is represented as a graph $G(V, E)$, where each node represents intersections, and each edge represents a road segment. 
Each edge $(i,j) \in E$ $(i, j \in V)$ is associated with a weight $c(i,j)$ which is a travel cost (can be either time or distance) from $i$ to $j$.
% TODO: define path in road network
The shortest path cost $d(s,t)$ is defined as the minimal cost path connecting $s$ and $t$. With our approach, APART, time and distance can be converted from one to the other.

\begin{definition} [Ride Request]
\label{def:req}
A ride request r can be represented as $\left\langle s, e, w, \epsilon, F \right\rangle$ consisting of a starting point $s \in V$ and an end point $e \in V$. Each request also specifies $w$ as the maximum time the rider can wait after making a request and the maximal detour $\epsilon\cdot d(s, e)$ the rider can afford . In addition, a rider's profile $F: \delta d \rightarrow \left[ 0, 1 \right] $, specifies the relative discount in exchange for an incurred detour of $\delta d$.
%the ridesharing fare she would accept based on the extra detour and the shortest possible trip.
\end{definition}

Upon the acceptance of a request, \fname assigns it to a driver.

\begin{definition} [Driver]
A driver v is represented as $\left\langle L, n, g \right\rangle$ where $L$ is the list of ride requests assigned to $v$, and $n$ is the maximum number of requests v can accept at any point in time. A driver also has a profile $g: d \rightarrow \$ $ which specifies the monetary cost of v driving a distance d while servicing its assigned requests.
\end{definition} 

\begin{definition} [Schedule]
Given a set $L$ with n requests, a schedule $S= \left\langle x_1, \cdots, x_{2n} \right\rangle$ is an ordered sequence of pick-up and drop-off points for these request, where for each $r_i \in L$, $r_i.s$ precedes $r_i.e$ in $S$. 
\end{definition}

A schedule is \textit{valid} for a driver \textit{v}, if it satisfies the following conditions:

\begin{itemize}
\item The riders' waiting time constraint: for any request $r_i$, the waiting time from the time the request is made until $v$ arrives at $r_i.s$ should be less than $r_i.w$.
\item The driver's capacity constraint: the number of riders in the vechile cannot exceed the total capacity $n$.
\item Detour constraint: the maximum distance of every rider's trip should be less than $(1+\epsilon) \cdot d(r_i.s, r_i.e)$.
\item The driver's and all riders' (the new rider and the those already in the vehicle) monetary constraints (See \cref{sec:pricing})
\end{itemize}

The driver follows the sequence of picking up and dropping off riders. The schedule changes over time as riders are serviced (picked-up/dropped-off) and new requests are added to the schedule. In fact, adding a new request to a schedule can re-order some requests that already exist in the schedule. For example, when a new request $r_3$ arrives, the initial schedule of $\left\langle s_1, s_2, e_1, e_2 \right\rangle$ can be reordered to $\left\langle s_1, s_2, s_3, e_2, e_3, e_1 \right\rangle$, where rider 1 is dropped off after rider 2.

\begin{definition} [Matching]
Assuming we have a set of drivers V and a set of requests R, we call $M \subset V \times R$ a matching if for each $r \in R$ there is at most one $v \in V$ such that $\left( v, r \right) \in M$. We call $\left( v, r \right) \in M$ \emph{a match}.
\end{definition}

\noindent In a matching $M$, for every driver $v$, there exists a \textit{valid} schedule $S_v$, such that $(v, r_i) \in M \implies r_i.s \in s_v \wedge r_i.e \in S_v$ (or simply $r_i \in S_v$). 

In \cref{sec:pricing}, we define a generic \textit{pricing model} where given a driver and its schedule, the pricing model computes the final fare each rider has to pay, the income of the driver and the ride-sharing platform's profit. Subsequently, we can define the \textit{ride-sharing} problem as follows:

\begin{definition} [Ride-sharing Problem]
Given a set of ride requests R and a set of drivers V, the goal of the ride-sharing problem is to find a matching M between R and V such that the revenue of M is maximized.
\end{definition}

%\dingedit{Justify that our objective function is different with minimizing the travel cost.} \moedit{If we have space at the end, we can give a simple contrary example to prove this.}

%\subsection{Problem Complexity}

%In this section, we perform a brief complexity analysis of the ride-sharing problem. First we show that the problem is NP-Hard. Subsequently, we define the \textit{online} version of the problem and prove there is no online algorithm that can generate an n-approximation solution for the ride-sharing problem.

%\begin{theorem}
%The Ride-Sharing problem is NP-Hard.
%\end{theorem}

%\begin{proof}
%For a given solution of the ride-sharing problem, one can compute the fare of all riders by finding the shortest possible route for their trips and subsequently, the length of the detour incurred to their route. Similarly, we can find the total traveled distance of the riders and compute their income and hence, compute the revenue of the platform. Therefore, the problem is \textit{verifiable} in polynomial time.\\
%Next, we show that the Vehicle Routing Problem (VRP) \cite{Dantzig59} is reducible to the ride-sharing problem in polynomial time. For that we correspond each node in the VRP problem to a request such that the pick-up location of the request is the location of the node and the drop-off point of the request is the central depot in the VRP problem. We also assume, regardless of their detour, every rider pays a constant $f$ and every driver is rewarded $i$ per unit of length. Assuming there are n nodes in the VRP problem, there is a solution of size $K$ for VRP, if and only if there is a rider to driver matching in the ride-sharing problem that generates a revenue of $n.f - K$. Therefore, VRP is reducible to the ride-sharing problem in polynomial time which concludes the proof.
%\end{proof}

The ride-sharing problem is NP-Hard since the Vehicle Routing Problem (VRP) \cite{Dantzig59} is reducible to the ride-sharing problem in polynomial time. A globally optimal solution to the ride-sharing problem can be achieved when a Clairvoyant exists which knows what requests are going to be submitted to the framework at what time and also has the knowledge of which drivers are going to be available, in advance. However, in this paper we study the \textit{online} version of the problem, i.e., the framework has no knowledge regarding future requests and incoming requests have to be matched with drivers as soon as they are submitted to the framework. The optimality of online algorithms are usually analyzed using \textit{competitive ratio} \cite{Sleator85}, i.e., an algorithm $\mathcal{A}$ is called \textit{c-competitive} for a constant $c > 0$, if and only if, for any input $\mathcal{I}$ the result of $\mathcal{A}(\mathcal{I})$ is at most $c$ times worst than the globally optimal solution. In the following we show no online algorithm can achieve a good competitive ratio for ride-sharing problem.

\begin{theorem}
\label{th:comp_ratio}
There does not exist a deterministic online algorithm for the ride-sharing problem that is \textit{c-competitive} ($c > 0$). 
\end{theorem}

\begin{proof}
Suppose there exists an algorithm $\mathcal{A}$ that is \textit{c-competitive}. We assume there exist a Clairvoyant which knows every decision $\mathcal{A}$ makes. For simplicity, we assume there is only one driver at point $(0, 0)$. The input starts with $r_1$ with a pick-up location at $(w, 0)$ and $r_2$ with pick-up location at $(-w, 0)$ (we assume all requests have a maximum wait time of $w$). The algorithm can make three choices for the driver. (1) move toward $r_1$, (2) move towards $r_2$ and (3) stay still. If choice 1 is selected, the Clairvoyant can generate the input such that at time $t = 1$, $n$ more request are submitted with pick-up location at $(-w-1, 0)$ and drop-off locations similar to $r_2$. Similar arguments can be made if choice 2 or 3 are selected by the algorithm. A globally optimal solution can complete $n+1$ requests while $\mathcal{A}$ can at most complete one request. By adding more drivers far away in a similar situation, the Clairvoyant can make $\mathcal{A}$'s solution unboundedly worse than the optimal solution. Therefore, we contradicted the assumption that $\mathcal{A}$ is \textit{c-competitive}.
\end{proof}

%%% TODO: given a rider and the current schedule, calculate the extra profit of inserting a new request.

%Properties of the above definition, can we guarantee the extra profit should be positive?

