\section{Problem Definition}

In this section, we define the terminologies and a generic pricing model used in the paper.  At the end we give a formal definition of the problem under consideration.
%Next we define rider request, driver, price models (user and rider profiles), and our optimization goals.

\subsection{Basic Concepts}
The road network is represented as a graph $G(V, E)$, where each node represents intersections, and each edge represents a road segment. 
Each edge $(u,v) \in E$ $(u, v \in V)$ is associated with a weight $c(u,v)$ which is a travel cost (can be time or distance) from $u$ to $v$.
% TODO: define path in road network
The shortest path cost $d(s,t)$ is defined as the minimal cost paths connecting $s$ and $t$. In this paper, time and distance can be converted from one to the other.

\begin{definition} [Ride Request]
\label{def:req}
A ride request r can be represented as $\left\langle s, e, w, f \right\rangle$ consisting of a starting point $s \in V$ and an end point $e \in V$. Each request also specifies $w$ as the maximum time the rider can wait after making a request. In addition, a rider's profile $f: \delta d \rightarrow \left[ 0, 1 \right] $, specifies the relative discount in exchange for an incurred detour of $\delta d$.
%the ridesharing fare she would accept based on the extra detour and the shortest possible trip.
\end{definition}

Upon the acceptance of a request, \fname assigns it to a driver.

\begin{definition} [Driver]
A driver v is a human, driving a vehicle on the road network and is represented as $\left\langle TR, n, g \right\rangle$ where $TR$ is the list of v's assigned ride requests and $n$ is the maximum number of requests v can accept at any point in time. A driver also has a profile $g: d \rightarrow \$ $ which specifies the monetary cost of v driving a distance d while servicing its assigned requests.
\end{definition} 

\begin{definition} [Schedule]
Given a set $TR$ with n requests, a schedule $s= \left\langle x_1, \cdots, x_{2n} \right\rangle$ is an ordered sequence of pickup and delivery points for these request, where for each $r_i \in TR$, $r_i.s$ preceds $r_i.e$ in $s$. 
\end{definition}

A schedule is \textit{valid} for driver \textit{v}, if it satisfies the following conditions:

\begin{itemize}
\item waiting time constraint
\item capacity constraint
\item profit constraint, which is the price constraint based on rider and driver profiles. \moedit{need a more formal definition for this}
\end{itemize}

The driver will follow the sequence of picking up and dropping off riders. The schedule changes over time as riders are picked-up/dropped-off and new requests are added to the schedule. In fact, adding a new request to a schedule, may re-order some of the request that already exist in the schedule.

\begin{definition} [Matching]
Assuming we have a set of Drivers V and a set of Requests R, we call $M \subset V \times R$ a matching if for each $r \in R$ there is at most one $v \in V$ such that $\left( v, r \right) \in M$. We call $\left( v, r \right) \in M$ a \emph{match} and say $r$ has been matched to $v$.
\end{definition}

\noindent In a matching $M$, for every driver $v$, there exists a valid schedule $s_v$, such that $(v, r_i) \in M \implies r_i.s \in s_v \wedge r_i.e \in s_v$ (or simply $r_i \in s_v$). 

In \cref{sec:pricing} we define a generic \textit{Pricing Model} where given a driver and its schedule, the pricing model will compute the final fare each rider has to pay, the income of the driver and the ride-sharing platform's profit. Subsequently, we can define the \textit{Ride-Sharing} problem as follows:

\begin{definition} [Ride-Sharing Problem]
Given a set of ride requests R and a set of drivers V, the goal of the Ride-Sharing problem is to find a matching M between R and V such that the revenue of M is maximized.
\end{definition}

\dingedit{Justify that our objective function is different with minimizing the travel cost.} \moedit{If we have space at the end, we can give a simple contrary example to prove this.}

\subsection{Cost Analysis}

\begin{theorem}
The Ride-Sharing problem is NP-Complete.
\end{theorem}

\begin{theorem}
There does not exist a deterministic online algorithm that is $c-competetive$ ($c > 0$). 
\end{theorem}

%%% TODO: given a rider and the current schedule, calculate the extra profit of inserting a new request.

%Properties of the above definition, can we guarantee the extra profit should be positive?

