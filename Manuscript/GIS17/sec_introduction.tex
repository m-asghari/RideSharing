\section{Introduction}
\label{sec:intro}

Traffic congestions, high costs of car ownership and greenhouse emissions are just a few problems that have made road transportation one of the main challenges of our era. Ride-sharing is an alternative transportation model which helps mitigate the negative effects of road transportation. According to \cite{Uber15}, in only one month, UberPool reduced individual rides in San Francisco for about 674,000 miles which accounts for more than 13,000 gallons of gasoline and about 120 metric tons of CO2 emissions. In addition to well-known ride-sharing platforms (e.g., Lyft, Uber, DiDi), other applications like Instacart and AmazonFresh, which use the same model as ride-sharing platforms, can also benefit from a real-time ride-sharing framework.

Due to its economical, environmental and social benefits, many researchers have studied ride-sharing platforms \cite{Kamar09,Kleiner11,Huang14,Cheng14,Ma15,Cici15,Shen16}. Many former studies \cite{Huang14,Ma15,Cici15} focus on improving the efficiency of the passenger-to-driver assignments by minimizing the total traveled distance which does not always satisfy the monetary constraints of passengers and drivers \cite{Asghari16}. Furthermore, none of the proposed approaches in these studies can scale due to their centralized architecture.

Auction methods have been effectively used for assignment problems in dynamic multi-agent environments \cite{Lagoudakis04,Mehta05}. The main advantage of auction methods are their simplicity and the fact that they allow for decentralized implementation. Consequently, in recent years, ride-sharing platforms have been studied in the context of auction methods \cite{Kamar09,Cheng14,Shen16}. One of the key requirements of such platforms is the ability to perform a real-time \emph{on-line} assignment of passengers to drivers as soon as a ride request is received. Most of the previous studies do not support on-line assignments\cite{Kamar09,Kleiner11,Cheng14}.

In \cite{Asghari16}, we proposed a ride-sharing platform, named APART, with a pricing model and a passenger-to-driver assignment algorithm. The main objective of APART is to maximizes the platform provider's profits, without compromising the passengers' and drivers' monetary incentives (i.e., either charging the passengers more or paying the drivers less). This is achieved by introducing the concept of \emph{profiles} as a mechanism for both parties to report their monetary constraints to the platform. The assignment algorithm guarantees these constraints are met when assigning a passenger to a driver. One drawback of the pricing model of APART is that drivers have an incentive to misreport their profiles, in order to increase their own income. For example, at certain times during the day when the demand (i.e., ride requests) to supply (i.e., available drivers) ratio is high, there will be less competition among drivers for each ride request. In such demand-driven markets, the drivers can increase their reported costs without significantly reducing their chance of getting matched with a passenger.

In this paper, first we present a statistical analysis that proves the drivers' dominant strategy in APART's pricing model is to misreport their profiles. That is, if a driver can estimate the number of available drivers in the vicinity of a new ride request then he can exploit this information to misrepresent his profile and still bid competitively. Subsequently, we present a latent space transition model (LSTM) which the drivers can utilize to predict the number of available drivers in every location at any point in time. We build this model by learning the transitional pattern of ride requests from historical data. We model the historical data as a set of spatial documents and utilize a \emph{Latent Dirichlet Allocation (LDA)}\cite{Blei03} approach to learn the parameters of the transitional pattern.

To address the problem of misreporting profiles, we present a new pricing mechanism for APART based on the second-price auction scheme with a reserved price (SPARP). We show that SPARP is both \emph{individually rational} (i.e., the drivers are incentivized to participate in the platform) and \emph{truthful} (i.e., the dominant strategy of drivers is to report their true profiles). We exploit the concept of \emph{reserved prices} in auction methods to prevent the platform from losing too much profit as a result of the second-price scheme and show how the value of the reserved price can be computed for different requests in APART.

We conducted extensive experiments on a large scale New York City taxi dataset. We compare the prediction accuracy of LSTM with a state-of-the-art approach \cite{Zhang14}. We show that our model can learn the transition patterns more accurately and hence, results in a 50\% more accurate prediction. Furthermore, we compare the utility of the drivers in a first-price auction scheme when a portion of the drivers bid untruthfully. We show that bidding untruthfully in a first-price auction scheme can result in up to 25\% higher income for drivers. Finally, we show that unlike a typical second-price auction scheme, by using reserved price in our mechanism, we can prevent the framework from losing too much profit.

The remainder of this paper is organized as follows. In \Cref{sec:background}, we briefly review the key components of APART that are relevant to this paper. In \Cref{sec:bidding}, we show that the dominant strategy of drivers in APART is to misreport their profiles. We present our latent space transition model in \Cref{sec:stmodel} which the drivers can utilize to predict the number of available drivers. We introduce our new pricing mechanism in \Cref{sec:mechanism}. In \Cref{sec:experiments}, we report the experiment results. \Cref{sec:related} reviews the related works and we conclude the paper in \Cref{sec:conclusion}.